\section{Sign languages}
\paragraph{}
Sign languages are complex natural languages used by deaf/mute people to communicate. Just like spoken languages, they are vary greatly depending on the region and culture in which they are used. In fact, there are hundreds of different sign languages used around the world, each with their own unique grammar and vocabulary.
\paragraph{}
Despite the complexity of sign languages, they can be broken down into smaller units, such as signs, hand shapes, and movements. Sign languages typically use a combination of these units to form words and sentences. For example, \ac{asp} uses hand shapes, movements, and facial expressions to convey meaning.
\paragraph{}
Sign languages are not just visual representations of spoken languages; they are unique and independent languages with their own syntax, grammar, and vocabulary. Recognizing and understanding them is therefore crucial for effective communication between hearing and deaf communities. In recent years, there has been increasing interest in developing technology to aid sign language recognition and translation.
\paragraph{Algerian Sign Language}
\ac{asp}, is a unique sign language used by the deaf community of Algeria. Technically, it is a visual-gestural language that lacks any defined grammar or syntax, making it difficult for those who don’t know the language to understand it. Therefore, the deaf community of Algeria is often excluded from basic communication. This has caused many deaf Algerians to go without access to education, health care, employment opportunities, and other basic rights. The lack of access to basic communication is especially detrimental to those living in remote areas of Algeria where the only people they can communicate with are other members of the deaf community. Fortunately, recent efforts have been made to teach sign language, both in formal and informal settings. The goal of these efforts is to give deaf Algerians access to education, health services and, eventually, gainful employment. This may help to reduce discrimination against the deaf community within Algeria. It is also hoped that, by teaching ALS, more people will be exposed to the unique language and culture of the deaf Algerians, and ultimately become more understanding and accepting of their community.