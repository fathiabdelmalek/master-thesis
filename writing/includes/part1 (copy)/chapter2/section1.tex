\section{Introduction}
In this chapter, we provide an overview of the related works in the field of sign language recognition (SLR). Sign language recognition is an important task that has gained much attention in recent years due to its potential to help improve communication between deaf and hearing individuals. The goal of SLR is to automatically recognize signs and gestures made by a person using a sign language. This is a challenging task due to the complexity and variability of sign languages, as well as the limitations of the sensing technology used to capture the movements.

The objective of this chapter is to review the state of the art in SLR, focusing on the different methods and techniques used to recognize sign language. We begin by discussing the different types of sign languages and their characteristics. We then present an overview of the different techniques used in SLR, including computer vision-based methods, data glove-based methods, and sensor-based methods. Finally, we conclude with a summary of the challenges and open problems in the field, highlighting the gaps in the existing literature and motivating the need for further research.