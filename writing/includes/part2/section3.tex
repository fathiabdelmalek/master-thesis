\section{Dataset collection and preprocessing}
\paragraph{}
In this section, we provide an overview of the dataset used for training and evaluating the \ac{slr} system implemented with the TAKALEM Gloves. Due to time constraints and limited resources, we did not construct our own "\ac{asp} dataset." Instead, the dataset used in this project is the "ASL-Sensor-Dataglove Dataset", which is publicly available for use. The dataset consists of sensor readings captured during the performance of \ac{asl} gestures.
\paragraph{}
The "ASL-Sensor-Dataglove" Dataset contains a comprehensive collection of \ac{sl} gestures, including 40 different signs derived from the \ac{asl} dictionary, it includes 26 letters of the alphabet and 14 commonly used words in \ac{sl}. The dataset was collected from a group of 25 subjects, comprising 19 males and 6 females. Each subject performed each sign gesture a total of 10 times, resulting in a dataset of 10,000 records. Therefore, there are 250 records available for each gesture, considering the repetitions from each subject. The dataset was recorded using a dataglove equipped with various sensors, capturing important hand movements and finger positions during the performance of each gesture.
\paragraph{}
To ensure consistency and usability, each gesture recording has a duration of one second and a half. Therefore, each gesture is represented by 150 consecutive rows in the CSV file, and each CSV file contains 1500 rows, representing a total of 1,500,000 rows in the entire dataset. This large-scale dataset enables robust training and evaluation of sign language recognition models.
\paragraph{}
The dataset is made publicly available for researchers and practitioners in the field of sign language recognition. It can be accessed through the following link: "ASL-Sensor-Dataglove" Dataset. The dataset owners have graciously made it accessible for public use, facilitating further advancements in sign language recognition research and development.
\begin{table}
	\centering
	\caption{Summary of the ASL-Sensor-Dataglove Dataset}
	\label{tab:dataset-summary}
	\begin{tabular}{|l|l|}
		\hline
		\textbf{Dataset Size} & 10,000 records \\
		\hline
		\textbf{Number of Signs} & 40 \\
		\hline
		\textbf{Subjects} & 25 (19 males, 6 females) \\
		\hline
		\textbf{Records per Sign} & 250 \\
		\hline
	\end{tabular}
\end{table}
\begin{table}
	\centering
	\caption{Selected Columns in the "ASL-Sensor-Dataglove" Dataset}
	\label{tab:column-used}
	\begin{tabular}{|l|l|c|}
		\hline
		\textbf{Sensor Type} & \textbf{Description} & \textbf{Number of Columns} \\
		\hline
		Flex Sensors & Measures finger bend & 5 \\
		\hline
		Gyro & Measures orientation and angular velocity & 3 \\
		\hline
		Acceleration & Measures acceleration along each axis & 3 \\
		\hline
	\end{tabular}
\end{table}
\paragraph{}
Additionally, the dataset contained additional columns for timestamp and user identification. These columns helped in tracking the temporal aspects of the gestures and associating them with the respective subjects. However, for the purposes of our \ac{dl} model, we primarily focused on the sensor data columns mentioned above.
\paragraph{}
Overall, the "ASL-Sensor-Dataglove" Dataset serves as a valuable resource for \ac{slr} research. Its availability to the public encourages collaboration and promotes the development of advanced models and algorithms in the field. The dataset's comprehensive nature, with 40 different signs and extensive recordings, allows for in-depth analysis and robust evaluation of \ac{slr} systems.