\renewcommand{\abstractname}{Résumé}
\begin{abstract}
	\paragraph{}
	La reconnaissance de la langue des signes joue un rôle crucial pour faciliter la communication entre les personnes malentendantes et les personnes normales. Nous voulons fabriquer des gants intelligents pour reconnaître la langue des signes algérienne (ASP) et la traduire en parole à l'aide d'algorithmes d'apprentissage en profondeur. Cependant, l'absence d'un ensemble de données ASP et les contraintes de temps limitées posent des défis dans le développement d'un ensemble de données personnalisé. Dans cette étude, un ensemble de données existant en langue des signes américaine (ASL) est utilisé et un modèle de mémoire à long terme (LSTM) est utilisé pour la classification des signes. Le modèle LSTM atteint une précision impressionnante de 95,03\% pour la classification au niveau des caractères et de 94,63\% pour la classification au niveau des mots. Remarquablement, le modèle présente des erreurs minimes dans les classifications de caractères et de mots, principalement en raison de la sélection du signe le plus fréquemment prédit à partir d'un pool de 150 prédictions par geste. Certes, ces résultats démontrent l'efficacité du modèle.
	
	\vspace{0.5cm}
	
	\providecommand{\keywords}[1]
	{
		\small	
		\textbf{\textit{Keywords---}} #1
	}
	
	\keywords{La langue des signes, ASP, ASL La reconnaissance de la langue des signes, Apprentissage en profondeur, LSTM}
\end{abstract}
