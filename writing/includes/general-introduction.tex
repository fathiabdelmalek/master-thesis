\chapter*{General Introduction}
\paragraph{}
This report presents the results and findings of our research on \ac{slr}. The objective of our study is to develop a \ac{dl} model that can accurately interpret and recognize \ac{sl} gestures. The ultimate goal is to bridge the communication gap between individuals who are fluent in \ac{sl} and those who are not, thereby fostering inclusivity and accessibility for the deaf community.
\paragraph{}
In this report, we provide a comprehensive overview of our research methodology, experiments, and results. The report is structured as follows:
\begin{itemize}
	\item \textbf{Chapter 1: Introduction} - This chapter provides an introduction to the research topic, outlines the objectives of the study, and explains the organization of the report.
	
	\item \textbf{Chapter 2: Dataset and Preprocessing} - Here, we present the ASL dataset used in our study, along with the preprocessing steps undertaken to prepare the data for model training. We describe the data collection process, the composition of the dataset, and any data cleaning or augmentation techniques employed.
	
	\item \textbf{Chapter 3: Methodology} - In this chapter, we detail the architecture and implementation of our deep learning models for both word and character recognition. We explain the choice of model architecture, the training procedure, and any optimization techniques utilized.
	
	\item \textbf{Chapter 4: Experimental Results} - Here, we present the evaluation results of our models. We analyze the performance metrics, including accuracy, precision, recall, and F1 score, and provide visual representations of the confusion matrices and classification reports.
	
	\item \textbf{Chapter 5: Discussion} - This chapter discusses the insights gained from the experimental results. We interpret the findings, identify strengths and limitations of our approach, and suggest potential areas for improvement and future research.
	
	\item \textbf{Chapter 6: Conclusion} - Finally, we summarize the key findings and contributions of our study. We reflect on the achievements and limitations of our research and outline the potential applications and impact of SLR systems for Arabic sign language.
\end{itemize}
\paragraph{}
By following this structured approach, we aim to provide a comprehensive understanding of our research process and outcomes. The subsequent chapters delve into the details of the dataset, methodology, and experimental results, offering insights into the effectiveness and applicability of our deep learning models for Arabic sign language recognition.
\paragraph{}
Through this report, we aim to contribute to the existing body of knowledge on SLR and advocate for the development of robust and accurate SLR systems that cater to the specific needs of the Arabic sign language community. We believe that our research findings have the potential to facilitate communication and promote inclusivity for individuals with hearing impairments.
