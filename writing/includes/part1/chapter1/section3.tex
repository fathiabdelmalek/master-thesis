\section{Sign language recognition}
\subsection{Definition}
\paragraph{}
\ac{slr} is the process of interpreting and translating the gestures, movements, and facial expressions of \ac{sl} into written or spoken language. It involves capturing, processing, and analyzing data from various sensors and devices such as gloves and cameras.
\paragraph{}
The task of \ac{slr} is a challenging one due to the complexity and variability of sign languages. They are rich and expressive, and there are many different sign languages used around the world, each with their own unique vocabulary, grammar, and syntax. Moreover, sign languages are not universal, meaning that a sign used in one language may have a completely different meaning in another language.
\paragraph{}
Despite these challenges, significant progress has been made in the field of \ac{slr} in recent years, thanks to advances in sensor technology, computer vision, \ac{ml} and \ac{dl}. Researchers have proposed a wide range of approaches to tackle the problem of \ac{slr}, including rule-based systems, template matching, \ac{hmm}, \ac{ann}, and \ac{dl} methods.
\paragraph{}
In recent years, \ac{dl}-based methods, particularly \ac{cnn} and \ac{rnn}, have shown promising results in \ac{slr}, achieving state-of-the-art performance on several benchmark datasets. These methods can learn meaningful representations of the data directly from raw input, which makes them well-suited to complex and dynamic data like \ac{sl}.
\subsection{\ac{slr} applications and use}
\paragraph{}
\ac{slr} technology has the potential to empower deaf and hard-of-hearing individuals by enabling them to communicate more effectively with the wider community. Moreover, it can be used in other areas alongside basic use, here are the main areas of use of \ac{slr} technology.
\subsubsection{\ac{sl} translation}
\paragraph{}
\ac{sl} translation is one of the primary applications of \ac{slr} technology. By capturing and analyzing \ac{sl} gestures, \ac{slr} systems can convert them into written or spoken language, enabling effective communication between deaf and hearing individuals. This technology can be utilized in various contexts, such as educational settings, customer service centers, healthcare facilities, and public institutions. For example, in education, \ac{slr} translation can support deaf students by providing real-time \ac{sl} interpretation during lectures, ensuring they have access to the same educational content as their hearing peers. In customer service, \ac{slr} translation can facilitate communication between deaf customers and service representatives, improving accessibility and customer satisfaction. Additionally, \ac{slr} translation can be integrated into translation apps or devices, allowing deaf individuals to communicate with individuals who do not understand \ac{sl}, bridging the communication gap.
\subsubsection{Virtual and augmented reality}
\paragraph{}
\ac{slr} technology can enhance the immersive experience of virtual and augmented reality environments by incorporating \ac{slr} capabilities. In VR/AR applications, \ac{slr} enables users to interact with the virtual world using \ac{sl} gestures, making the experience more intuitive and inclusive for deaf users. For instance, in a \ac{vr} game, \ac{slr} can recognize and interpret \ac{sl} commands as input, allowing players to control their characters or perform actions using \ac{sl} gestures. In \ac{ar}, \ac{slr} can be used to overlay real-time \ac{sl} translations onto the user's field of view, enabling seamless communication between deaf and hearing individuals in \ac{ar} scenarios. This integration of \ac{slr} in \ac{vr}/\ac{ar} not only enhances entertainment experiences but also opens up new possibilities in training simulations, remote collaboration, and interactive storytelling.
\subsubsection{Research and linguistics}
\paragraph{}
\ac{slr} technology plays a crucial role in linguistic research and the study of sign languages. By capturing and analyzing \ac{sl} data, \ac{slr} systems provide valuable insights into the structure, grammar, and syntax of sign languages. Researchers can use \ac{slr} to examine the linguistic patterns and variations within \ac{sl} communities, contributing to the documentation and preservation of sign languages. \ac{slr} can also assist in studying the cognitive aspects of \ac{sl} processing and acquisition. Furthermore, \ac{slr} can be used to create \ac{sl} corpora and databases, which serve as valuable resources for linguistic analysis and comparison across different \ac{sl} systems. Overall, \ac{slr} technology empowers researchers and linguists to delve deeper into the intricate nature of sign languages, leading to a better understanding and appreciation of deaf culture and communication.