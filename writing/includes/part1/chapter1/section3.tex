\section{Sign language recognition}
Sign language recognition is the process of interpreting and translating the gestures, movements, and facial expressions of sign language into written or spoken language. It involves capturing, processing, and analyzing data from various sensors and devices such as gloves, cameras, and accelerometers, among others.

The task of sign language recognition is a challenging one due to the complexity and variability of sign languages. Sign languages are rich and expressive, and there are many different sign languages used around the world, each with their own unique vocabulary, grammar, and syntax. Moreover, sign languages are not universal, meaning that a sign used in one language may have a completely different meaning in another language.

Despite these challenges, significant progress has been made in the field of sign language recognition in recent years, thanks to advances in sensor technology, computer vision, and machine learning. Researchers have proposed a wide range of approaches to tackle the problem of sign language recognition, including rule-based systems, template matching, Hidden Markov Models (HMMs), Artificial Neural Networks (ANNs), and Deep Learning (DL) methods.

In recent years, DL-based methods, particularly Convolutional Neural Networks (CNNs) and Recurrent Neural Networks (RNNs), have shown promising results in sign language recognition, achieving state-of-the-art performance on several benchmark datasets. These DL-based methods can learn meaningful representations of the data directly from raw input, which makes them well-suited to complex and dynamic data like sign language.

In the next section, we will provide an overview of some of the related works in the field of sign language recognition, including some of the most significant contributions in the literature.