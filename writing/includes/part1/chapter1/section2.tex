\section{Sign languages}
Sign languages are complex natural languages used by deaf and hard-of-hearing people to communicate. Just like spoken languages, sign languages vary greatly depending on the region and culture in which they are used. In fact, there are hundreds of different sign languages used around the world, each with their own unique grammar and vocabulary.

Despite the complexity of sign languages, they can be broken down into smaller units, such as signs, handshapes, and movements. Sign languages typically use a combination of these units to form words and sentences. For example, American Sign Language (ASL) uses handshapes, movements, and facial expressions to convey meaning. Similarly, British Sign Language (BSL) uses handshapes, movements, and body posture to convey meaning.

Sign languages are not just visual representations of spoken languages; they are unique and independent languages with their own syntax, grammar, and vocabulary. Recognizing and understanding sign languages is therefore crucial for effective communication between hearing and deaf communities. In recent years, there has been increasing interest in developing technology to aid sign language recognition and translation.