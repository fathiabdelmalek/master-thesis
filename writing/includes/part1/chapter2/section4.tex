\section{Conclusion}

Sign language recognition is a challenging task, but it has made significant progress in recent years. Deep learning methods have shown to be very effective for SLR, and they are likely to continue to improve the accuracy of SLR systems in the future.

There are a number of challenges that still need to be addressed in SLR. One challenge is the variability of sign language production. Another challenge is the complexity of sign language grammar. Finally, there is a lack of large-scale annotated datasets for SLR.

Despite these challenges, there is a lot of potential for SLR. SLR systems can be used to improve communication between deaf and hearing people, and they can also be used for education and translation. As more research is conducted on SLR, it is likely that these systems will become more accurate and widely used.
