\section{Methods of Sign Language Recognition}

There are a number of different methods that have been used for sign language recognition. Some of the most commonly used methods include:

* **Handcrafted features:** These methods use hand-crafted features to represent sign language videos. The features are then used to train a classifier to recognize signs.

\begin{itemize}
	\item Example of a handcrafted feature: the number of fingers that are extended in a sign.
\end{itemize}

* **Hidden Markov models (HMMs):** HMMs are a statistical model that can be used to represent the temporal dynamics of sign language videos. HMMs have been used successfully for SLR.

\begin{itemize}
	\item Example of an HMM: an HMM that models the sequence of handshapes, locations, and movements in a sign.
\end{itemize}

* **Convolutional neural networks (CNNs):** CNNs are a type of deep learning model that can be used to extract features from sign language videos. CNNs have been shown to be very effective for SLR.

\begin{itemize}
	\item Example of a CNN: a CNN that is trained to extract features from the spatial dimensions of a sign language video.
\end{itemize}

* **Recurrent neural networks (RNNs):** RNNs are another type of deep learning model that can be used to represent the temporal dynamics of sign language videos. RNNs have been shown to be effective for SLR.

\begin{itemize}
	\item Example of an RNN: an RNN that is trained to predict the next handshape in a sign, given the previous handshapes.
\end{itemize}

The choice of method will depend on the specific application of the SLR system. For example, if the SLR system is being developed for use with a small dataset, then a handcrafted features approach might be a good choice. However, if the SLR system is being developed for use with a large dataset, then a deep learning approach might be a better choice.

The accuracy of SLR systems has improved significantly in recent years. This improvement is due to the development of new deep learning methods. Deep learning methods have been shown to be very effective for SLR, and they are likely to continue to improve the accuracy of SLR systems in the future.
