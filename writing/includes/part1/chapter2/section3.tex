\section{Synthesis}
In this section, we provide a synthesis of the various methods and techniques for sign language recognition that have been discussed in section 2. We classify these methods into three categories: vision-based, sensor-based, and hybrid approaches.

Vision-based approaches rely on analyzing video data of the signer's hand gestures to recognize signs. These methods typically use computer vision techniques such as background subtraction, motion detection, and feature extraction to identify and track the signer's hands and fingers. However, vision-based approaches may face challenges such as lighting conditions, occlusion, and variability in hand shapes.

Sensor-based approaches, on the other hand, use wearable sensors to capture data on the signer's hand movements and postures. These sensors can include accelerometers, gyroscopes, flex sensors, and force sensors. Sensor-based approaches have the advantage of being more robust to lighting conditions and occlusion than vision-based approaches. However, they require careful calibration and placement of sensors on the glove or other wearable device.

Hybrid approaches combine both vision-based and sensor-based methods to improve accuracy and robustness. For example, some systems use vision-based methods to detect the signer's hand location and then switch to sensor-based methods for gesture recognition. Other systems use sensor-based methods for fine-grained gesture recognition and vision-based methods for coarse-grained gesture recognition.

Overall, there is no single best approach for sign language recognition, and the choice of method depends on the specific requirements and constraints of the application.