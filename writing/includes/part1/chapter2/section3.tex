\section{Synthesis}
\paragraph{}
\ac{slr} is a complex task that requires a combination of computer vision techniques, \ac{ml} algorithms, and domain-specific knowledge. Researchers have explored different methodologies and strategies to tackle this challenge. The synthesis of related works allows us to identify common trends, key techniques, and advancements in the field.
\paragraph{}
Several studies have focused on the utilization of computer vision techniques for hand gesture detection and tracking. Techniques such as background subtraction, hand region segmentation, and hand shape analysis have been employed to extract relevant visual features from input videos or images. These features serve as the basis for subsequent classification or recognition processes.
\paragraph{}
Furthermore, \ac{ml} algorithms play a pivotal role in \ac{slr}. Various approaches, including traditional \ac{ml} methods and \ac{dl} techniques, have been explored. Traditional \ac{ml} algorithms such as \ac{svm}, decision trees, and random forests have shown promising results in certain scenarios. These algorithms require handcrafted features and extensive feature engineering, which can be time-consuming and resource-intensive.
\paragraph{}
\ac{dl} models, on the other hand, have demonstrated remarkable performance in \ac{slr} tasks. \ac{cnn}s and \ac{rnn}s have been widely adopted. \ac{cnn}s excel in capturing spatial features and have proven effective for hand gesture recognition. \ac{rnn}s, particularly \ac{lstm} networks, have been successful in capturing the temporal dynamics of \ac{sl} gestures.
\paragraph{}
From the analysis of related works, it is evident that \ac{dl} models, particularly \ac{cnn}s and \ac{rnn}s, have emerged as powerful tools for \ac{slr}. \ac{cnn}s excel in capturing spatial features from visual inputs, while \ac{rnn}s effectively model the temporal dynamics of \ac{sl} gestures. The integration of these models has shown promising results, leading to improved accuracy and real-time performance.
\paragraph{}
To address the scarcity of labeled \ac{sl} datasets, researchers have explored various strategies, such as data augmentation, transfer learning, and semi-supervised learning. Data augmentation techniques generate additional training samples by applying transformations, deformations, or adding noise to existing data. Transfer learning allows leveraging pre-trained models on large-scale datasets and fine-tuning them on \ac{sl} data. Semi-supervised learning combines a small amount of labeled data with a larger pool of unlabeled data to enhance model performance.
\paragraph{}
Furthermore, the availability of labeled \ac{sl} datasets has been a crucial factor in the advancement of the field. Researchers have developed dedicated datasets for training and evaluating \ac{slr} models. However, the limited size and diversity of these datasets pose challenges in training robust models that generalize well across different sign languages and variations.
\paragraph{}
The evaluation and benchmarking of \ac{slr} methods are crucial to assess their effectiveness and compare different approaches. Researchers have utilized various evaluation metrics such as accuracy, precision, recall, and F1 score to measure the performance of their models. Datasets specifically designed for \ac{slr}, such as RWTH-BOSTON-50 and Jochen Triesch's dataset, have been widely used for evaluation purposes.
\paragraph{}
Despite the progress made, there are still several challenges and opportunities for future research. One key challenge is the robustness of \ac{slr} models in handling variations in hand shape, orientation, and articulation. Developing models that can adapt to individual user variations and different \ac{sl} dialects is an ongoing area of research.
\paragraph{}
Another avenue for exploration is the integration of multimodal inputs, such as combining visual information from gloves or cameras with other sensory inputs like speech or haptic feedback. This multimodal approach has the potential to enhance the accuracy and usability of \ac{slr} systems, providing more comprehensive and natural communication interfaces.