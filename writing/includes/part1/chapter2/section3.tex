\section{Evaluation Metrics and Performance Analysis}
\paragraph{}
Evaluation metrics play a crucial role in measuring the performance of \ac{slr} systems. There are various evaluation metrics used for this purpose, including accuracy, precision, recall, F1-score, and confusion matrix.
\paragraph{}
Accuracy is the most commonly used metric that measures the percentage of correctly classified signs by the model. Precision measures how many of all predicted signs were correct while Recall measures how many actual signs were recognized correctly.
\paragraph{}
F1-score evaluates both precision and recall simultaneously to provide a more comprehensive understanding of model performance. It considers false positives and false negatives as well.
\paragraph{}
The Confusion Matrix is another effective way to evaluate \ac{slr} models. It provides information about true positive (correctly recognized) and false negative (incorrectly not-recognized) rates.
\paragraph{}
It's important to note that selecting appropriate evaluation metrics depends on the specific requirements of each application or dataset being used for training/testing purposes. Therefore choosing an optimal metric should be done with caution based on its suitability for application-specific goals