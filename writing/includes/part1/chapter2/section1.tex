\section{Introduction}
\paragraph{}
In this section, we provide an introduction to the topic of \ac{slr} and its importance in facilitating communication for the deaf and hard of hearing community. We present an overview of the goals and objectives of this study, highlighting the significance of developing accurate and efficient \ac{slr} systems. Additionally, we outline the specific objectives of this research and provide an overview of the subsequent sections in this chapter.
\subsection{Background}
\paragraph{}
In this subsection, we provide background information on sign languages, their unique characteristics, and their importance as a means of communication for individuals with hearing impairments. We discuss the complexity of sign languages and the challenges associated with their recognition, emphasizing the need for advanced technologies to aid in real-time sign language interpretation.
\subsection{Motivation}
\paragraph{}
In this subsection, we discuss the motivation behind the development of sign language recognition systems. We highlight the limitations of traditional communication methods for individuals with hearing impairments and the potential impact that accurate and efficient sign language recognition can have on their daily lives. We also discuss the growing interest in using machine learning and artificial intelligence techniques to enhance sign language recognition capabilities.
\subsection{Objectives}
\paragraph{}
In this subsection, we outline the specific objectives of this research project. We identify the main goals and targets that we aim to achieve, such as designing a wearable sign language recognition system, developing robust machine learning models, and evaluating the system's performance through experiments and user studies. We also highlight the potential applications and benefits of the proposed system in real-world scenarios.
\subsection{Overview of the Chapter}
\paragraph{}
In this subsection, we provide an overview of the subsequent sections in this chapter. We briefly describe the content and organization of each section, outlining the main topics and discussions that will be covered. This serves as a roadmap for the reader, giving them a clear understanding of what to expect in the rest of the chapter.
\paragraph{}
By following this structure, you can introduce the context, motivation, objectives, and the overall organization of Chapter 2 in your thesis. Feel free to customize the content and add more details based on the specific focus and requirements of your research.