\section{Introduction}
\subsection{Sign Language Recognition: An Overview}
\paragraph{}
Sign language recognition refers to the process of interpreting and understanding sign language gestures to facilitate communication between individuals who are deaf or hard of hearing and those who are not proficient in sign language. Sign languages are complex visual languages with distinct grammar, syntax, and cultural variations. Recognizing and translating sign language gestures in real-time present significant challenges due to the intricacies and nuances involved.
\paragraph{}
The development of robust and accurate sign language recognition systems has gained increasing attention in recent years due to the potential impact on inclusive education, healthcare accessibility, smart home integration, and emergency communication. By bridging the communication gap between signers and non-signers, these systems contribute to creating an inclusive environment for individuals with hearing impairments.
\subsection{Motivation for Exploring Related Works}
\paragraph{}
Exploring related works in sign language recognition is essential for several reasons. Firstly, understanding the existing literature and research provides valuable insights into the state of the field, including advancements, methodologies, and limitations. This knowledge enables researchers to build upon previous findings and avoid duplicating efforts.
\paragraph{}
Secondly, analyzing related works helps identify gaps and challenges in current approaches to sign language recognition. By understanding the limitations of existing methods, researchers can propose innovative solutions to address these limitations and enhance the overall accuracy, speed, and usability of recognition systems.
\paragraph{}
Furthermore, advancements in sign language recognition technology have the potential to transform various sectors. In education, sign language recognition systems can assist in teaching sign language to non-signers, facilitating inclusive classrooms and promoting sign language literacy. In healthcare settings, these systems enable effective communication between healthcare providers and patients with hearing impairments, enhancing the quality of care and patient outcomes.
\paragraph{}
Moreover, integrating sign language recognition into smart homes and devices enhances accessibility and convenience for individuals who rely on sign language as their primary mode of communication. Additionally, in emergency situations where verbal communication may be challenging or impossible, sign language recognition systems can play a crucial role in ensuring effective communication and timely assistance.
\subsection{Objectives of the Chapter}
\paragraph{}
This chapter aims to provide a comprehensive overview of sign language recognition by exploring related works in the field. The specific objectives are as follows:
\subsubsection{Review and Analyze Existing Methodologies}
\paragraph{}
The chapter will examine various methodologies and techniques used in sign language recognition, including vision-based methods, data glove-based methods, and hybrid approaches. By analyzing these methodologies, their strengths, limitations, and performance, researchers can gain insights into the different approaches used in the field.
\subsubsection{Explore Datasets for Training and Evaluation}
\paragraph{}
The chapter will investigate existing sign language datasets utilized for training and evaluating sign language recognition systems. It will discuss the characteristics of representative datasets, such as size, diversity, annotation methods, and challenges associated with dataset collection and annotation.
\subsubsection{Evaluate Performance Metrics}
\paragraph{}
The chapter will delve into the evaluation metrics commonly used to assess the performance of sign language recognition systems. It will discuss metrics such as accuracy, precision, recall, and F1 score, highlighting their relevance and interpretation in the context of sign language recognition.
\subsubsection{Identify Advancements and Future Directions} 
\paragraph{}
By analyzing related works, the chapter aims to identify advancements, trends, and potential research directions in sign language recognition. This includes identifying areas of improvement, emerging technologies, and challenges that require further exploration.