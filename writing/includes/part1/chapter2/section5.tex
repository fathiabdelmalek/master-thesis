\section{Synthesis}
\paragraph{}
Synthesis in \ac{slr} refers to the process of combining different approaches and techniques to achieve better results. With the increasing interest in \ac{dl}, researchers have incorporated \ac{ann} models such as \ac{lstm} into \ac{slr} systems.
\paragraph{}
In addition to using \ac{lstm}, researchers also employ other methods such as \ac{hmm}, \ac{svm}, and \ac{cnn}. By combining these various techniques, it is possible to improve accuracy rates and reduce error rates in sign language recognition.
\paragraph{}
Moreover, datasets play a crucial role in synthesis since they provide a means for training and testing models. Researchers use publicly available datasets or create their own annotated corpus of data for specific purposes.
\paragraph{}
Evaluation metrics are used to measure the performance of synthesized systems. Metrics like precision, recall and F1-score help researchers assess the effectiveness of their approach.
\paragraph{}
The development of synthesized systems has great potential for improving communication between hearing-impaired individuals and those who do not know \ac{sl}. Future research studies on this topic will continue exploring new ways of integrating different techniques that enable accurate recognition algorithms with faster processing times than current state-of-the-art solutions can offer.